% This is a comment

% Here, we load IEEEtran class with option conference
\documentclass[conference]{IEEEtran}

% The region after \documentclass and before \begin{document} is called preamble.
% You can load packages and define format of the document here.

% The \input{...} command read another tex file.
% This is particularly useful when dealing with large document.
% For example, you may want put each section of the paper in a separated tex file.
% Here, we put most of preamble definitions in preamble.tex file.
\usepackage{color}
\usepackage[table,dvipsnames]{xcolor}

\usepackage{microtype}
\usepackage{etoolbox}
\usepackage{float}
\usepackage{placeins}
\usepackage{balance}
\usepackage{footmisc}

\usepackage{amsmath}
\usepackage{amssymb}
\let\emptyset\varnothing%
\usepackage{amsfonts}
\usepackage{mathrsfs}
\usepackage{latexsym}
\usepackage[ruled,vlined]{algorithm2e}
\usepackage{amsthm}
\usepackage{thmtools,thm-restate}
\declaretheorem[style=plain]{axiom}
\declaretheorem[style=definition]{definition}
\declaretheorem[style=remark]{example}
\declaretheorem[style=plain]{lemma}
\declaretheorem[style=plain]{theorem}
\declaretheorem[style=remark]{remark}

\usepackage{caption}
\usepackage{subcaption}
\usepackage{graphicx}
\usepackage[mode=buildnew]{standalone}

\usepackage{array}
\usepackage{booktabs}
\usepackage{multirow}

\usepackage{listings}

\usepackage{hyperref}
\usepackage[capitalise]{cleveref}
\hypersetup{%
    unicode,
    psdextra,
    hidelinks,
    bookmarksnumbered=true,
    bookmarksopen=true,
    bookmarksopenlevel=3,
    plainpages=false,
    pdfstartview={XYZ null null 1},
    pdfpagemode={UseOutlines},
    pdfpagelayout={OneColumn}
}

\usepackage[inline]{enumitem}
\setlist{noitemsep,partopsep=0pt,topsep=0pt}
\newlist{inlineenum}{enumerate*}{1}
\setlist[inlineenum]{label=(\roman*),ref=(\roman*)} % chktex 36

\usepackage[%
    style=ieee,
    dashed=false,
    doi=false,
    isbn=false,
    giveninits=true,
    maxnames=10,
]{biblatex}
\addbibresource{ref.bib}



\title{An Example of \LaTeX~Usage}
\author{John Doe}

\begin{document}

% typeset the title
\maketitle

\begin{abstract}
  This document shows how to use \LaTeX~to typeset an academic paper.
\end{abstract}

\section{Basic Syntax}

It does not matter whether
you enter one or several
spaces        after a word.

An empty line starts a new
paragraph.

You can force a line break without starting a new paragraph. Like this: \\
There will be a line break before this sentence.

You could also specify the vertical space after the line break. \\ [2ex]
It creates a line break with vertical space equaling to 2 times of the height of character `x'.

\section{Special Characters and Symbols}

The following commands are used to typeset special characters:
\# \$ \% \^{} \& \_ \{ \} \~{} \textbackslash

Quotation marks should be typeset as following: `single quoted text' and ``double quoted text''.

There are four kinds of dashes in \LaTeX{}.
\begin{itemize}
  \item hyphen: part-time.
  \item en-dash: pages 1--10.
  \item em-dash: yes---or no?
  \item minus sign: $0$, $1$, and $-1$.
\end{itemize}

Ellipsis can be typeset as following: New York, Tokyo, Budapest, \ldots

\section{Font Face \& Size}

\subsection{Font Face Commands}

\textrm{roman}
\textsf{sans serif}
\texttt{typewriter}
\textmd{medium}
\textbf{bold face}
\textup{upright}
\textit{italic}
\textsl{slanted}
\textsc{small caps}
\emph{emphasized}
\textnormal{document font}

\subsection{Font Size Commands}

{\tiny tiny font}                \\
{\scriptsize very small font}    \\
{\footnotesize quite small font} \\
{\small small font}              \\
{\normalsize normal font}        \\
{\large large font}              \\
{\Large large font}              \\
{\LARGE very large font}         \\
{\huge huge}                     \\
{\Huge largest}

\section{Space and Alignment}

Force an unbreakable space using tilde symbol. e.g.~these~spaces~are~unbreakable. They cannot be broken by line break or page break.

In addition to the normal space, a breakable space can be inserted using backslash following by a space. For example, there are three spaces\ \ \ inside this sentence.

You can also specify the space like\hspace{5em}this.

\vspace{5ex}

The above command create a vertical space.

To align the text.

\begin{center}
  text to be centered
\end{center}

\begin{flushleft}
  text to be flushed left
\end{flushleft}

\begin{flushright}
  text to be flushed right
\end{flushright}

\section{List Structures}

\begin{enumerate}
  \item You can nest the list
        environments to your taste:
        \begin{itemize}
          \item But it might start to
                look silly.
          \item[-] With a dash.
        \end{itemize}
  \item Therefore remember:
        \begin{description}
          \item[Stupid] things will not
                become smart because they are
                in a list.
          \item[Smart] things, though,
                can be presented beautifully
                in a list.
        \end{description}
\end{enumerate}

You can custom the list using \texttt{enumitem} package.

\begin{itemize}[label=-]
  \item A
  \item B
\end{itemize}

List environments can be inlined. The following \texttt{inlineenum} environment is defined in the \texttt{preamble.tex} file.
\begin{inlineenum}
  \item A
  \item B
\end{inlineenum}

\section{Math}

Inline math: \(\sum_i^k i\)

Display math: \[\sum_i^k\]

Math using the \texttt{equation} environment shown in \cref{eqn:1}.
\begin{equation}
  E = mc^2 \label{eqn:1}
\end{equation}

\Cref{eqn:2,eqn:3} use the \texttt{align} environment.
\begin{align}
  B' & =-\nabla \times E, \label{eqn:2}         \\
  E' & =\nabla \times B - 4\pi j, \label{eqn:3}
\end{align}

Disable numbering for some equation.
\begin{align}
  a & = b + c \nonumber \\
    & = d + e
\end{align}

Keep equations aligned across text.
\begin{align*}
  F & = f_1+f_2+f_3+...+f_n
  \intertext{can be written as} % use \intertext{} to insert text while keep equations aligned.
  F & = \sum_1^n{f_i}
\end{align*}

A list of equations without the alignment.
\begin{gather*}
  \cos(2\theta) = \cos^2\theta-\sin^2\theta \\
  \lim\limits_{x \to \infty} \exp(-x) = 0 \\
  a \bmod b \\
  x \equiv a \pmod{b} \\
  \log{(N)} \\
  \operatorname{arg\,max}_a f(a) = \operatorname*{arg\,max}_b f(b) \\
  n^{22} \\
  f(n) = n^5 + 4n^2 + 2 |_{n=17} \\
  \sum_{i=1}^{n} i \\
  \lim_{x \to \infty} \frac{1}{x} \\
  \frac{n!}{k!(n-k)!} = \binom{n}{k} \\
  \sqrt{2} \\
  \sqrt[n]{1+x+x^2+x^3+\dots+x^n} \\
  \left(\frac{x^2}{y^3}\right) \\
  P\left(A=2\middle|\frac{A^2}{B}>4\right) \\
  \left\{\frac{x^2}{y^3}\right\}
\end{gather*}

Typeset matrices.

\begin{gather*}
  \begin{pmatrix}
    a & b & c \\
    d & e & f \\
    g & h & i
  \end{pmatrix} =
  \begin{bmatrix}
    a & b & c \\
    d & e & f \\
    g & h & i
  \end{bmatrix}
\end{gather*}

\begin{gather*}
  \left(\begin{array}{cc|c}
    a & b & c \\
    d & e & f
  \end{array}\right)
\end{gather*}

Matrices can be embed inside another matrix.

\begin{gather*}
  \begin{pmatrix}
    \begin{pmatrix}
      a_{11} & a_{12} \\
      a_{21} & a_{22} \\
    \end{pmatrix}
           & 0      & \cdots \\
    0      &
    \begin{pmatrix}
      b_{11} & b_{12} \\
      b_{21} & b_{22} \\
    \end{pmatrix}
           & \cdots          \\
    \vdots & \vdots & \ddots
  \end{pmatrix}
\end{gather*}

Cases:

\begin{gather*}
  f(x) = \left\{
  \begin{array}{ll}
    x & \text{if } x > 0, \\
    0 & \text{otherwise}.
  \end{array}\right.
\end{gather*}

\begin{gather*}
  f(n) =
  \begin{cases}
    n/2      & \quad \text{if } n \text{ is even} \\
    -(n+1)/2 & \quad \text{if } n \text{ is odd}
  \end{cases}
\end{gather*}

\begin{table}[t]
  \centering
  \begin{tabular}{ll}
    \toprule
    \textbf{Code}          & \textbf{Output}                     \\
    \midrule
    \verb|\mathnormal{...}| & $\mathnormal{ABCDEF~abcdef~123456}$ \\
    \verb|\mathrm{...}| & $\mathrm{ABCDEF~abcdef~123456}$     \\
    \verb|\mathit{...}| & $\mathit{ABCDEF~abcdef~123456}$     \\
    \verb|\mathbf{...}| & $\mathbf{ABCDEF~abcdef~123456}$     \\
    \verb|\mathsf{...}| & $\mathsf{ABCDEF~abcdef~123456}$     \\
    \verb|\mathtt{...}| & $\mathtt{ABCDEF~abcdef~123456}$     \\
    \verb|\mathfrak{...}| & $\mathfrak{ABCDEF~abcdef~123456}$   \\
    \verb|\mathcal{...}| & $\mathcal{ABCDEF}$                  \\
    \verb|\mathbb{...}| & $\mathbb{ABCDEF}$                   \\
    \bottomrule
  \end{tabular}
  \caption{Math Fonts}\label{tab:math-fonts}
\end{table}

\begin{table}[t]
  \centering
  \begin{tabular}{llr}
    \toprule
    \multicolumn{2}{c}{Item}             \\
    \cmidrule(r){1-2}
    Animal    & Description & Price (\$) \\
    \midrule
    Gnat      & per gram    & 13.65      \\
              & each        & 0.01       \\
    Gnu       & stuffed     & 92.50      \\
    Emu       & stuffed     & 33.33      \\
    Armadillo & frozen      & 8.99       \\
    \bottomrule
  \end{tabular}
  \caption{An example of table}\label{tab:example}
\end{table}

\begin{figure*}[t]
  \centering
  \begin{subfigure}[b]{0.45\linewidth}
    \includegraphics[width=\linewidth]{example-image-a}
    \caption{Example Image A}\label{fig:example:a}
  \end{subfigure}
  ~%
  \begin{subfigure}[b]{0.45\linewidth}
    \includegraphics[width=\linewidth]{example-image-b}
    \caption{Example Image B}\label{fig:example:b}
  \end{subfigure}
  \caption{An example of figure}\label{fig:example}
\end{figure*}

\begin{figure}[t]
  \centering
  \includestandalone[width=0.8\linewidth]{./tikz-example} % without the `.tex` extension
  \caption{TikZ Figure in Article}\label{fig:tikz}
\end{figure}

\section{Figure \& Table}

\Cref{tab:math-fonts} lists a variety of fonts available in the math mode. \Cref{tab:example} shows another example of table.

\Cref{fig:example} consists of two figures: \cref{fig:example:a} and \cref{fig:example:b}. An figure drawn by TikZ in shown in \cref{fig:tikz}.

\section{Theorems}

\begin{definition}[Prime]
  A prime is a natural number greater than 1 that cannot be formed by multiplying two smaller natural numbers.
\end{definition}

\begin{theorem}[Euclid]\label{thm:euclid}
  For every prime $p$, there is a prime $p’>p$.
\end{theorem}

\begin{lemma}
  According to \cref{thm:euclid}, there are infinitely many primes.
\end{lemma}

% use restatable to repeat a theorem.
\begin{restatable}[Fermat's Last Theorem]{theorem}{fermantlast}\label{thm:fermantlast}%
  Diophantine Equation \(x^n + y^n = z^n\), where $x$, $y$, $z$, and $n$ are integers, has no nonzero solutions for $n>2$.
\end{restatable}

\section{Algorithms \& Source Code}

\begin{algorithm}[t]
  \caption{How to write algorithms}\label{alg:1}
  \KwData{this text}
  \KwResult{learn to write algorithm}
  initialization\;
  \While{not at end of this document}{
    read current\;
    \eIf{understand}{
      go to next section\;
      current section becomes this one\;
    }{
      go back to the beginning\;
    }
  }
\end{algorithm}

\begin{lstlisting}[language=Python,float=t,caption={A example of listing},label=lst:1]
def fib():
  a, b = 0, 1
  while 1:
    yield a
    a, b = b, a + b
\end{lstlisting}

\Cref{alg:1} and \cref{lst:1} show the example of pseudo algorithm and source code highlight.

\section{Cross-references}\label{sec:ref}

You can reference to \cref{sec:ref}. Or add some footnote~\footnote{Use \url{https://www.google.com} when you encounter problems in \LaTeX.}

\section{Bibliography}

You can cite a paper like this~\cite{shannon1948}.
Or cite multiple papers at the same time~\cite{merkle1989,roussopoulos1995}.
It is also useful to cite the authors. For example, \citeauthor*{roussopoulos1995} proposed a method to process nearest neighbor queries.

To add an item in the reference list but without direct citation using \verb|\nocite{}| command.

\nocite{latextutoiral}

\section*{Acknowledgment}

Use \verb|\section*{}| command to create a section without numbering. It is commonly to be used in the section of acknowledgment.

% Use the following command offered by biblatex to print the reference
\printbibliography

% To create appendix, it is often to use \appendix command offered by \usepackage{appendix}.
% However, we use the \appendices command here, which is provided by IEEEtran class.
\appendices

\section{Proof of Theorem~\ref{thm:fermantlast}}

% repeat the theorem
\fermantlast*

\begin{proof}
  There is a proof that was too large to fit in the margin.
\end{proof}

\end{document}
